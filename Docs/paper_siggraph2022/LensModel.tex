%%
\section{Differentiable Lens Model}
\label{sec:optics}
As in most existing complex lens system,  the refraction are usually generated by either spericial lenses or asperical lenses.
Throughout the rest of this paper, we consider rotationally symmetrical lens profile designs which can be rotational symmetry facilitates manufacturing using turning machines.

In addition, our lens model can be easily applied to rotationally asymmetrical profiles like the expression 
of Zernik basis or even diffractive optics, and offer a new way to jointly optimize the refractive optics and diffractive optics.
\vspace{-3pt}
\subsection{Asperical Lens}
%
 The phase of a lens describes the delay of the incident wave phase introduced by the lens element, at the lens plane. The geometrical (ray) optics model, commonly used in computer graphics, models light as rays of photon travel instead of waves. 
%\fh{we should also visualize this. maybe Fig.~\ref{fig:optics_design} can be adjusted.} 
This model ignores diffraction, e.g. for light passing through a narrow slit. Although being an approximation to physical optics, ray optics still can provide an intuition: the perpendiculars 
%\gw{I don't think this word exists. please reformulate - not clear.}
%\fh{it's the noun version of the adjective perpendicular: https://dictionary.cambridge.org/dictionary/english/perpendicular} 
to the waves can be thought of as rays, and, vice versa, phase intuitively describes the relative delay of photons traveling along these rays to the lens plane, as illustrated with red lines in Fig.~\ref{fig:optics_design}. Hence, the phase of a thin lens is its height profile multiplied with the wave number and the refractive index~\cite{hecht1998hecht,goodman2005introduction}.  

 %We design the proposed lens system by optimizing a lens phase profile for a sparse set of on-axis and off-axis directions, corresponding to different focus areas on the image plane at different offsets from the image center. 
We design the proposed lens by first specifying an ideal phase profile
for perfect, spatially invariant PSFs over the full FOV, i.e., mapping
incident rays from one direction to one single point. Because it will
turn out intractable to manufacture this ideal lens, we propose an
aperture partitioning strategy as an approximation. The deviation of this
partitioned phase profile to the ideal profile is a
large low-frequency component which is independent of the incident
angle. Together with the peak-component, which preserves local
contrast over the full FOV, these two components make up the desired spatially invariant dual-mixture PSF. 

 To specify the ideal phase profile $\phi(r,\omega_i)$ for an incident ray direction $i$, and radial position $r$, see Figure~\ref{fig:optics_design}, we assume a physical aperture size $D$, focus distance $f$, and set: 

Aspheric lenses have been traditionally defined with the surface profile (sag) given by: 
\begin{equation}
\label{eq:sag}
Z(s) = \frac{cs^2}{1+\sqrt{1-(1+ \kappa )c^2s^2}} + \Sigma_{k=1} c_ks^{2k+2}
\end{equation}
%
 where $Z$ represents here is the sag of surface parallel to the optical axis,  $s$
 represents the radial distance from the optical axis, $c=1/R$ is the curvature or inverse of radius
 $R$. 
 $c_k$ represents the high order asperical coefficients. When the asperical 
 coefficients all set to zeros, the resulting lens surface becomes a conic $\kappa$.
 
 
 For  this  ideal lens profile, we define the output angle as:
%
 
\begin{equation}
\label{eq:targe_phase_2}
	\theta(r,\omega_i) = \mathrm{arctan} \left( \frac{\rho(\omega_i) - r}{f} \right),
\end{equation}
%
since the ideal lens design maps the incident rays from one direction $\omega_i$ to a single point with spatial position $\rho(\omega_i)$ on the image plane.

Next,  by inserting Eq.~\ref{eq:targe_phase_2} into Eq.~\ref{eq:targe_phase_1},  we derive the target phase $\phi$ as:
%
\begin{align}
%\begin{equation}
\label{eq:targe_phase_3}
%\begin{aligned}
\phi(r,\omega_i) &= -k \left[ r \cdot \mathrm{sin}\omega_i - \int_{0}^{r} \frac{\rho(\omega_i)-r_1}{\sqrt{f^2 + (\rho(\omega_i)-r_1)^2}} dr_1 \right] \\
&= -k\left[ r \cdot \mathrm{sin}\omega_i + \sqrt{f^2 +(\rho(\omega_i)-r)^2} - \sqrt{f^2 + \rho(\omega_i)^2} \right]. \nonumber
%\end{aligned}
%\end{equation}
\end{align}
%
 The ideal phase profile from Eq.~\ref{eq:targe_phase_3} is visualized
in Figure~\ref{fig:optics_design} (right). We observe a drastic
variation when approaching larger incident angles. In other words, the
same position on the lens aperture would need to realize different phases for
different incident angles, which is not physically realizable with
thin plate optics. 
%

 
\subsection{Aperture Partitioning}
%
Realizing the ideal phase profile is intractable to manufacture over the full aperture, as illustrated by the large angular deviations needed in off-axis region in Figure~\ref{fig:optics_design}.
To overcome this challenge,
we split the aperture into multiple sub-regions, and
assign each sub-region to a different angular interval, similar to prior work~\cite{levin20094d,zhu2013design} for refractive optics. 
 We note that this concept is also closely related to specializing optics depending on the incident ray direction in light field imaging~\cite{ng2005light}, for example, tailoring optical aberrations for digital correction~\cite{ng2012correction}. 
Specifically, we introduce a \emph{virtual aperture} $\mathcal{A}(r,\omega_i) =
circ[r-\mathrm{\nu(\omega_i)}]$ to partition the incident light bundle
of each direction into a peak component that we optimize for, while treating out-of-aperture 
components as out-of-focus blur. 
Here, $circ[\cdot]$ is a function representing a circular aperture, $\nu(\omega_i)$ indicates the axial center of the virtual
aperture subject to the $i^{\text{th}}$ incident ray direction. With this aperture partitioning, we optimize for the phase profile solving the following optimization problem: 
%we reasonably partition the incoming light bundle of each view into two components 
%Thus, our goal is to enforce a compromised high intensity peak of PSFs for all views, %we reasonably partition the incoming light bundle into two components. For this %purpose, we introduce a \emph{virtual aperture} $\mathcal{A}(x,\omega)=circle(x-d(\omega))$ and express the designing problem as the following optimization problem:    
\begin{equation}
\label{eq:lensOptim}
[\mathrm{\phi_0}(r),\mathrm{\rho},\mathrm{\nu}] = \mathop{\arg\min}_{\mathrm{\phi_0}(r),\mathrm{\rho,\mathrm{\nu}}}
\,\sum_{i=1}^{N} \|\mathcal{A}(r,\omega_i)(\phi_0(r)-\phi(r,\omega_i))\|_{2}^{2}. 
\end{equation}

Note that the virtual aperture is not a physical aperture
of the optical system, but is only  introduced as a conceptual partitioning in the lens optimization.
Figure~\ref{fig:float_aperture} shows the virtual apertures for uniformly sampled directions
superimposed on the real aperture.  For every direction, we optimize only
for the rays that pass through the  corresponding virtual apertures ; these
will be focused into a sharp PSF, while all other rays from the same
direction that miss $\mathcal{A}$ but pass through the full aperture
$D$ will be blurred and manifest as a low frequency ``haze'' in the measurement.

%% As we set constraints only on the \emph{virtual aperture}, the rays    %%
%% outside this aperture \emph{generally} considered out-of-focus,        %%
%% thereby can be reasonably assumed to form a large spot that is visible %%
%% as a low-frequency ''haze'' in the measurement.                        %%

%This is similar to the concept of aperture multiplexing but differs in that our virtual apertures have overlaps that we aim to compromise under a realistic effective aperture size.

 
\subsection{Fresnel Depth Profile Optimization} 
We solve the optimization problem from Eq.~\ref{eq:lensOptim} using
Zemax~\cite{geary2002introduction}.  While Eq.~\ref{eq:lensOptim} minimizes phase differences, Zemax interprets it as minimizing the optical path difference (OPD). 
Zemax allows us to piggy-back on a library of parameterized surface types, and directly
optimize a deep Fresnel lens profile ( a deeper micro-structure than regular 2$\pi$ modulation. ) instead of sequentially optimizing for the
phase and depth in a two-stage process.   We formulate the problem from Eq.~\ref{eq:lensOptim} using the multiple configuration function 
with the number of the configurations set to the discretized aperture directions (7 in this paper, uniformly sampled on half of the diagonal image size). We set the
size of each virtual aperture -- the effective aperture that contributes to focusing light bundles -- to one third
of the clear aperture.  As shown in
Figure~\ref{fig:float_aperture}, the center $\nu$ of the virtual
aperture for each direction along the clear aperture plane
can be modeled by shifting a stop along the optical
axis.  This allows us to optimize the location of the virtual aperture
by setting the stop position as an additional optimization
variable.  The merit (objective) function used in Zemax includes terms for
minimizing the wavefront (phase) error at each sampled direction, and
enforcing a desired effective focal length (EFL). 
We refer the reader to the supplementary document for additional details.

 
% 
\vspace{6pt}
\subsection{Aberration Analysis}
%
The optical aberrations of the proposed design have the following properties.
The chromatic variation is small because a deep Fresnel surface results in only small focal length differences in the visible wavelength region.
Off-axis variation (i.e. spatial intensity variation of PSFs 
across FOV) are small since we only control a part of light 
of each direction to focus into the sharp peak
(see Figures~2 and~7).

 
 
For each viewing direction, the PSF exhibits two components, a
high-intensity peak, which preserves local contrast, and a large
low-frequency component. We note that this property differs from 
 conventional spherical or aspherical singlets with the same
NA whose field curvature can be severe.  Although the low-frequency
PSF component reduces contrast, it does so
uniformly across the FOV.  In contrast to conventional single
  element optics, which have very poor contrast in regions far from
  the optical axis (required for wide-FOV imaging), it
  is this design which allows us to preserve the ability to detect
  some residual contrast, instead of completely losing contrast. 
%


